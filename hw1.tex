\documentclass{article}
\usepackage{geometry}
\usepackage{listings}
\usepackage{fontspec}
\usepackage{xeCJK}
\fontsize{48}{48}
\setCJKmainfont[ItalicFont={SimSun}, BoldFont={SimHei}]{SimSun}
\setCJKsansfont{SimHei}
\setCJKmonofont{FangSong_GB2312}
\setmainfont{Times New Roman}
\setsansfont{Times New Roman}[Scale=MatchLowercase]
\setmonofont{Times New Roman}[Scale=MatchLowercase]

\geometry{paper=a4paper}
\geometry{layout=a4paper}
\author{Silenuszhi}
\title{Verilog Experiment 1 for SOC student}

\begin{document}
\maketitle
\newpage{}
\section{Uart Excercise}
\paragraph{Deadline : Oct 10th 2016\\}
In this section, you need to write a uart transmit(tx) module, a uart receive(rx) module, and a frame receive module to unpack GPS \$GPGGA data, You should write the code and do the simulation, finally with a writeup in pdf send to xuechengbo@gmail.com
\paragraph{Must do:}
\subparagraph {a)!You must use a enhanced text editor,such as notepad++,vi,sublime,emacs,etc.Choose the one you like.But at least,the editor should has the code highlight.}
\subparagraph {b)!!!You must use clk/clock as the clock signal's name, use resetn/reset\_n/rst\_n/rstn as the reset signal's name,all reset should be active low in our lab.Besides,if a data signal is active low, you should add $\_n$ or $n$ or $\_b$ or $b$ at the end of the signal, $\_n$ is preferred}
\subparagraph {c)!!!!You should never use "signal1","signal2" as your signal name}
\subparagraph {d)!!You must choose Modelsim/VCS as your verilog compiler,and use Modelsim/Verdi to watch simulation waveform, or you will not get help from others in our lab. We do only use Modelsim and Synopsys tools}
\subparagraph {e)!!!You must write a script to compile the source file and do simulation.}
\paragraph{Should do:}
\subparagraph {a)The editor may also have auto complete and auto indent}
\subparagraph {b)Try a better font so you can distinguish 0 and O}
\subsection{Uart TX Module}
The port definition is given below, Find out what uart is and write a uart module.\\
0) clk is 100MHz\\
1) clk\_en signal is generated by a module called clk\_divisor,the signal is going to be asserted one clock period according to the baudrate 115200,The actual baudrate used in system could have an at most 3\% error.And you should generate the signal in your testbench\\
2) Datain and shoot signal means when tx\_module is not busy and shoot is asserted,then send the datain serial out, shoot must be one clock valid and datain is only valid at that period\\
3) Dart\_busy signal means whether the module is ready to accept a new shoot\\
4) As parameter, Verify\_on and Verify\_even set the verify method of the uart module\\
5) Write the code and testbench, do the simulation, and write a simple write up to explain what you have done 


\begin{lstlisting}[language=Verilog]
module uart_tx_op
  #(
    parameter VERIFY_ON = 1'b0,
    parameter VERIFY_EVEN = 1'b0
    )
   (
    input       clk_i,
    input       resetn_i,
    input       clk_en_i, 
    input [7:0] datain_i,
    input       shoot_i,
    output reg  uart_tx_o = 1'b1,
    output reg  uart_busy_o
    );
\end{lstlisting}
\subsection{Uart RX Module}
The port definition is given below.\\
1) Becareful every clk that dataout\_valid\_o asserts means a valid dataout\_o.\\
2) You can use the Uart Tx Module help you to verify this one.\\
3) Write the code and testbench, do the simulation, and write a simple write up to explain what you have done 
\begin{lstlisting}[language=Verilog]
module uart_rx
  #(
    parameter VERIFY_ON = 1'b0,
    parameter VERIFY_EVEN = 1'b0
    )
   (
    input            clk_i,
    input            clk_en_i,
    input            resetn_i,
    input            uart_rx_i,
    output reg       dataout_valid_o,
    output reg [7:0] dataout_o 
    );
\end{lstlisting}

\subsection{rx\_package module}
\paragraph{notation\\1) start of frame (sof)\\
2) end of frame (eof)}

\paragraph{scenario\\1) The actual gps module output signal is mixed several types of package,like \$GPGGA and \$GPZDA etc, we are going to pick up a specific one.\\
2) When we use custom frame, there will use an fixed length of frame, we are going to pick up the fixed frame with one module as well.}

\paragraph{Problems\\}
The module is capable of handle a uart receive data stream use sof+eof mode or sof+framecnt mode and pickup(fixed position and length), substitute some words(fixed position and length) and output the valid frame.\\
The port definition is given below:\\
1) If use sof+eof mode, set the EOFDETECTION to 1 \\
2) If use sof+framecnt mode, set the FRAMELENGTHFIXED to 1\\
3) At first the module is coded to connect to 2 fifo, so that the toggle signal can be used to know which fifo is writen to.\\
4) sub\_data\_valid is one clock period assert\\
5) Other parameter is clear enough\\
6) Write the code and testbench, do the simulation, and write a simple write up to explain what you have done 

\begin{lstlisting}[language=Verilog]
module rx_package
  #(
    parameter TOGGLE = 1'b1, //use pingpong buffer    
    parameter SOFLENGTH = 2, //sof length
    parameter SOFPATTERN = 16'hEB90,//sof pattern
    parameter EOFDETECTION = 1'b1,//eof detect or not
    parameter EOFLENGTH = 2,//eof length
    parameter EOFPATTERN = 16'h90EB,//eof pattern
    parameter FRAMELENGTHFIXED = 1'b1, //if 1,frame is fixed length
    parameter FRAMECNT = 64,//framecnt
    parameter SUB = 1'b1,//substitution
    parameter SUBPOS = 2,
    parameter SUBLENGTH = 8,//substitution length,
// if not used, leave unconnected, will gen a warning   
    parameter PICKPOS = 8,
    parameter PICKLENGTH = 3 //pick some bytes output
    )
   (
    input                         clk,
    input                         reset,
    input                         enable,

    input                         rx_data_valid,
    input [7:0]                   rx_data, 

    input                         sub_data_valid, 
    input [SUBLENGTH*8-1:0]       sub_data,
   
    output [PICKLENGTH*4-1:0]     pick_data,
    output reg                    pick_data_valid,
    output                        FIFO_clear,
   
    output reg [TOGGLE:0]         frame_datavld = 0, 
    output reg [7:0]              frame_data = 0,
    output reg [10:0]             frame_count = 0, 
    output reg [TOGGLE:0]         frame_interrupt = 0
    );
\end{lstlisting}





\section{code for clk\_divider}
\begin{lstlisting}[language=Verilog]
module clk_divider
  #(parameter DIVISOR = 6'd0)   
   (
    input      clk_i,
    input      resetn_i,
    output reg clk_en_o
    );

   reg [5:0]   clk_dividor = 0;   

   // clk divider
   always @ ( posedge clk_i or negedge resetn_i ) begin
      if(!resetn_i)begin
         /*AUTORESET*/
         // Beginning of autoreset for uninitialized flops
         clk_dividor <= 6'h0;
         clk_en_o <= 1'h0;
         // End of automatics
      end else begin
         if (clk_dividor != DIVISOR) begin
            clk_dividor <= clk_dividor + 1'b1;
            clk_en_o <= 1'b0;
         end else begin
            clk_dividor <= 6'h0;
            clk_en_o <= 1'b1;
         end
      end      
   end
endmodule
\end{lstlisting}

\end{document}